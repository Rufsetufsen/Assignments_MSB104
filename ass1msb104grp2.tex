% Options for packages loaded elsewhere
% Options for packages loaded elsewhere
\PassOptionsToPackage{unicode}{hyperref}
\PassOptionsToPackage{hyphens}{url}
\PassOptionsToPackage{dvipsnames,svgnames,x11names}{xcolor}
%
\documentclass[
  british,
  a4paper,
]{article}
\usepackage{xcolor}
\usepackage{amsmath,amssymb}
\setcounter{secnumdepth}{5}
\usepackage{iftex}
\ifPDFTeX
  \usepackage[T1]{fontenc}
  \usepackage[utf8]{inputenc}
  \usepackage{textcomp} % provide euro and other symbols
\else % if luatex or xetex
  \usepackage{unicode-math} % this also loads fontspec
  \defaultfontfeatures{Scale=MatchLowercase}
  \defaultfontfeatures[\rmfamily]{Ligatures=TeX,Scale=1}
\fi
\usepackage{lmodern}
\ifPDFTeX\else
  % xetex/luatex font selection
\fi
% Use upquote if available, for straight quotes in verbatim environments
\IfFileExists{upquote.sty}{\usepackage{upquote}}{}
\IfFileExists{microtype.sty}{% use microtype if available
  \usepackage[]{microtype}
  \UseMicrotypeSet[protrusion]{basicmath} % disable protrusion for tt fonts
}{}
\makeatletter
\@ifundefined{KOMAClassName}{% if non-KOMA class
  \IfFileExists{parskip.sty}{%
    \usepackage{parskip}
  }{% else
    \setlength{\parindent}{0pt}
    \setlength{\parskip}{6pt plus 2pt minus 1pt}}
}{% if KOMA class
  \KOMAoptions{parskip=half}}
\makeatother
% Make \paragraph and \subparagraph free-standing
\makeatletter
\ifx\paragraph\undefined\else
  \let\oldparagraph\paragraph
  \renewcommand{\paragraph}{
    \@ifstar
      \xxxParagraphStar
      \xxxParagraphNoStar
  }
  \newcommand{\xxxParagraphStar}[1]{\oldparagraph*{#1}\mbox{}}
  \newcommand{\xxxParagraphNoStar}[1]{\oldparagraph{#1}\mbox{}}
\fi
\ifx\subparagraph\undefined\else
  \let\oldsubparagraph\subparagraph
  \renewcommand{\subparagraph}{
    \@ifstar
      \xxxSubParagraphStar
      \xxxSubParagraphNoStar
  }
  \newcommand{\xxxSubParagraphStar}[1]{\oldsubparagraph*{#1}\mbox{}}
  \newcommand{\xxxSubParagraphNoStar}[1]{\oldsubparagraph{#1}\mbox{}}
\fi
\makeatother

\usepackage{color}
\usepackage{fancyvrb}
\newcommand{\VerbBar}{|}
\newcommand{\VERB}{\Verb[commandchars=\\\{\}]}
\DefineVerbatimEnvironment{Highlighting}{Verbatim}{commandchars=\\\{\}}
% Add ',fontsize=\small' for more characters per line
\usepackage{framed}
\definecolor{shadecolor}{RGB}{241,243,245}
\newenvironment{Shaded}{\begin{snugshade}}{\end{snugshade}}
\newcommand{\AlertTok}[1]{\textcolor[rgb]{0.68,0.00,0.00}{#1}}
\newcommand{\AnnotationTok}[1]{\textcolor[rgb]{0.37,0.37,0.37}{#1}}
\newcommand{\AttributeTok}[1]{\textcolor[rgb]{0.40,0.45,0.13}{#1}}
\newcommand{\BaseNTok}[1]{\textcolor[rgb]{0.68,0.00,0.00}{#1}}
\newcommand{\BuiltInTok}[1]{\textcolor[rgb]{0.00,0.23,0.31}{#1}}
\newcommand{\CharTok}[1]{\textcolor[rgb]{0.13,0.47,0.30}{#1}}
\newcommand{\CommentTok}[1]{\textcolor[rgb]{0.37,0.37,0.37}{#1}}
\newcommand{\CommentVarTok}[1]{\textcolor[rgb]{0.37,0.37,0.37}{\textit{#1}}}
\newcommand{\ConstantTok}[1]{\textcolor[rgb]{0.56,0.35,0.01}{#1}}
\newcommand{\ControlFlowTok}[1]{\textcolor[rgb]{0.00,0.23,0.31}{\textbf{#1}}}
\newcommand{\DataTypeTok}[1]{\textcolor[rgb]{0.68,0.00,0.00}{#1}}
\newcommand{\DecValTok}[1]{\textcolor[rgb]{0.68,0.00,0.00}{#1}}
\newcommand{\DocumentationTok}[1]{\textcolor[rgb]{0.37,0.37,0.37}{\textit{#1}}}
\newcommand{\ErrorTok}[1]{\textcolor[rgb]{0.68,0.00,0.00}{#1}}
\newcommand{\ExtensionTok}[1]{\textcolor[rgb]{0.00,0.23,0.31}{#1}}
\newcommand{\FloatTok}[1]{\textcolor[rgb]{0.68,0.00,0.00}{#1}}
\newcommand{\FunctionTok}[1]{\textcolor[rgb]{0.28,0.35,0.67}{#1}}
\newcommand{\ImportTok}[1]{\textcolor[rgb]{0.00,0.46,0.62}{#1}}
\newcommand{\InformationTok}[1]{\textcolor[rgb]{0.37,0.37,0.37}{#1}}
\newcommand{\KeywordTok}[1]{\textcolor[rgb]{0.00,0.23,0.31}{\textbf{#1}}}
\newcommand{\NormalTok}[1]{\textcolor[rgb]{0.00,0.23,0.31}{#1}}
\newcommand{\OperatorTok}[1]{\textcolor[rgb]{0.37,0.37,0.37}{#1}}
\newcommand{\OtherTok}[1]{\textcolor[rgb]{0.00,0.23,0.31}{#1}}
\newcommand{\PreprocessorTok}[1]{\textcolor[rgb]{0.68,0.00,0.00}{#1}}
\newcommand{\RegionMarkerTok}[1]{\textcolor[rgb]{0.00,0.23,0.31}{#1}}
\newcommand{\SpecialCharTok}[1]{\textcolor[rgb]{0.37,0.37,0.37}{#1}}
\newcommand{\SpecialStringTok}[1]{\textcolor[rgb]{0.13,0.47,0.30}{#1}}
\newcommand{\StringTok}[1]{\textcolor[rgb]{0.13,0.47,0.30}{#1}}
\newcommand{\VariableTok}[1]{\textcolor[rgb]{0.07,0.07,0.07}{#1}}
\newcommand{\VerbatimStringTok}[1]{\textcolor[rgb]{0.13,0.47,0.30}{#1}}
\newcommand{\WarningTok}[1]{\textcolor[rgb]{0.37,0.37,0.37}{\textit{#1}}}

\usepackage{longtable,booktabs,array}
\usepackage{calc} % for calculating minipage widths
% Correct order of tables after \paragraph or \subparagraph
\usepackage{etoolbox}
\makeatletter
\patchcmd\longtable{\par}{\if@noskipsec\mbox{}\fi\par}{}{}
\makeatother
% Allow footnotes in longtable head/foot
\IfFileExists{footnotehyper.sty}{\usepackage{footnotehyper}}{\usepackage{footnote}}
\makesavenoteenv{longtable}
\usepackage{graphicx}
\makeatletter
\newsavebox\pandoc@box
\newcommand*\pandocbounded[1]{% scales image to fit in text height/width
  \sbox\pandoc@box{#1}%
  \Gscale@div\@tempa{\textheight}{\dimexpr\ht\pandoc@box+\dp\pandoc@box\relax}%
  \Gscale@div\@tempb{\linewidth}{\wd\pandoc@box}%
  \ifdim\@tempb\p@<\@tempa\p@\let\@tempa\@tempb\fi% select the smaller of both
  \ifdim\@tempa\p@<\p@\scalebox{\@tempa}{\usebox\pandoc@box}%
  \else\usebox{\pandoc@box}%
  \fi%
}
% Set default figure placement to htbp
\def\fps@figure{htbp}
\makeatother



\ifLuaTeX
\usepackage[bidi=basic]{babel}
\else
\usepackage[bidi=default]{babel}
\fi
% get rid of language-specific shorthands (see #6817):
\let\LanguageShortHands\languageshorthands
\def\languageshorthands#1{}
\ifLuaTeX
  \usepackage[english]{selnolig} % disable illegal ligatures
\fi


\setlength{\emergencystretch}{3em} % prevent overfull lines

\providecommand{\tightlist}{%
  \setlength{\itemsep}{0pt}\setlength{\parskip}{0pt}}



 


\makeatletter
\@ifpackageloaded{caption}{}{\usepackage{caption}}
\AtBeginDocument{%
\ifdefined\contentsname
  \renewcommand*\contentsname{Table of contents}
\else
  \newcommand\contentsname{Table of contents}
\fi
\ifdefined\listfigurename
  \renewcommand*\listfigurename{List of Figures}
\else
  \newcommand\listfigurename{List of Figures}
\fi
\ifdefined\listtablename
  \renewcommand*\listtablename{List of Tables}
\else
  \newcommand\listtablename{List of Tables}
\fi
\ifdefined\figurename
  \renewcommand*\figurename{Figure}
\else
  \newcommand\figurename{Figure}
\fi
\ifdefined\tablename
  \renewcommand*\tablename{Table}
\else
  \newcommand\tablename{Table}
\fi
}
\@ifpackageloaded{float}{}{\usepackage{float}}
\floatstyle{ruled}
\@ifundefined{c@chapter}{\newfloat{codelisting}{h}{lop}}{\newfloat{codelisting}{h}{lop}[chapter]}
\floatname{codelisting}{Listing}
\newcommand*\listoflistings{\listof{codelisting}{List of Listings}}
\makeatother
\makeatletter
\makeatother
\makeatletter
\@ifpackageloaded{caption}{}{\usepackage{caption}}
\@ifpackageloaded{subcaption}{}{\usepackage{subcaption}}
\makeatother
\usepackage{bookmark}
\IfFileExists{xurl.sty}{\usepackage{xurl}}{} % add URL line breaks if available
\urlstyle{same}
\hypersetup{
  pdftitle={Assignment 1:},
  pdfauthor={Kristoffer Tufta},
  pdflang={en-GB},
  colorlinks=true,
  linkcolor={blue},
  filecolor={Maroon},
  citecolor={Blue},
  urlcolor={Blue},
  pdfcreator={LaTeX via pandoc}}


\title{Assignment 1:}
\usepackage{etoolbox}
\makeatletter
\providecommand{\subtitle}[1]{% add subtitle to \maketitle
  \apptocmd{\@title}{\par {\large #1 \par}}{}{}
}
\makeatother
\subtitle{Regional GDP Inequality in 4 Selected European Economies}
\author{Kristoffer Tufta}
\date{Monday 3 Nov, 2025}
\begin{document}
\maketitle


\begin{Shaded}
\begin{Highlighting}[]
\FunctionTok{library}\NormalTok{ (tidyverse)}
\end{Highlighting}
\end{Shaded}

\begin{verbatim}
-- Attaching core tidyverse packages ------------------------ tidyverse 2.0.0 --
v dplyr     1.1.4     v readr     2.1.5
v forcats   1.0.0     v stringr   1.5.1
v ggplot2   3.5.2     v tibble    3.3.0
v lubridate 1.9.4     v tidyr     1.3.1
v purrr     1.1.0     
-- Conflicts ------------------------------------------ tidyverse_conflicts() --
x dplyr::filter() masks stats::filter()
x dplyr::lag()    masks stats::lag()
i Use the conflicted package (<http://conflicted.r-lib.org/>) to force all conflicts to become errors
\end{verbatim}

\begin{Shaded}
\begin{Highlighting}[]
\FunctionTok{library}\NormalTok{ (PxWebApiData)}
\FunctionTok{library}\NormalTok{(readxl)}
\FunctionTok{library}\NormalTok{(purrr)}
\FunctionTok{library}\NormalTok{(dineq)}
\FunctionTok{library}\NormalTok{(psych)}
\end{Highlighting}
\end{Shaded}

\begin{verbatim}

Attaching package: 'psych'

The following objects are masked from 'package:ggplot2':

    %+%, alpha
\end{verbatim}

\section{Data Acquisition}\label{data-acquisition}

In this assignment we will use four selected countries data from
Eurostat to process it and analyse the sub-national GDP (gross domestic
product) and population data from the years 2000-2023. Eurostat serves
as the statistical office of the European Union, and their work is to
collect and provide statistics on EU countries, through reliable,
impartial and comparable data The countries in question are Germany,
Switzerland, Croatia and Ireland. In these datasets we encountered
missing values which we decided to keep. These NA, or missing data came
from different reason for each country.

\begin{itemize}
\tightlist
\item
  Germany which has the most observations, lacks data for GDP from the
  time 2023 in a lot of its regions. This can be because of late
  reporting of its data to Eurostat. There are also some missing data on
  population during 2000-2010 and a few other regions during 2000-2023
  which may be the emergence of new regions or change in their districts
  that require their own data.
\item
  Ireland lacks data from the early 2000 to 2011 in population due to
  changes in NUTS 3 level in their regions. When it comes to the GDP,
  Ireland only misses data from 2015-2017 in Mid-West and South-West.
  This was due to confidentiality concerns.
\item
  Croatia only have NA values on population but its spread by different
  regions. Same as Germany, here the lack of data can be explained by
  the changes in regions and districts, which may be the cause of the
  spread in NA values. It has also been shown that the NUTS2 regions
  have changed from 2007 to 2021, and the data has been reported using
  several different NUTS-definitions.
\item
  Switzerland is a non.EU, but EFTA country, and have not had a
  data-sharing agreement with Eurostat for NUTS3 GDP from 2000-2007,
  while in the 2008 the NUTS classification was updated and it was
  standardized across all regions. Switzerland also lacks the data from
  2022-2023 which may be they are waiting to finalize the data before
  releasing it.
\end{itemize}

We will then calculate the GDP per capita and explore regional
inequality through a EDA (exploratory data analysis).

\subsection{GDP per Region}\label{gdp-per-region}

In the code chunk below, we use the ``read\_excel''-function from the
package ``readxl'' to fetch the downloaded GDP data from Eurostat, and
put it away as a table into a function we have called ``raw\_econ''. We
then use the table stored in the function ``raw\_econ'' to create a tidy
dataset, which we have called ``tidy\_econ''.

\begin{Shaded}
\begin{Highlighting}[]
\CommentTok{\#Removing the metadata from the top and bottom by defining the range.}
\NormalTok{raw\_econ }\OtherTok{\textless{}{-}} \FunctionTok{read\_excel}\NormalTok{(}\StringTok{"./Data/GDP\_noFlag.xlsx"}\NormalTok{, }\AttributeTok{sheet =} \StringTok{"Sheet 1"}\NormalTok{, }\AttributeTok{range =} \StringTok{"A8:Y464"}\NormalTok{, }\AttributeTok{col\_types =} \StringTok{"text"}\NormalTok{)}
\CommentTok{\#Dropping the first row to align time with years}
\NormalTok{  raw\_econ }\OtherTok{\textless{}{-}}\NormalTok{ raw\_econ[}\SpecialCharTok{{-}}\DecValTok{1}\NormalTok{,]}
  \FunctionTok{names}\NormalTok{(raw\_econ)[}\DecValTok{1}\NormalTok{] }\OtherTok{\textless{}{-}} \StringTok{"Geo\_Labels"}
\FunctionTok{print}\NormalTok{(raw\_econ)}
\end{Highlighting}
\end{Shaded}

\begin{verbatim}
# A tibble: 455 x 25
   Geo_Labels     `2000` `2001` `2002` `2003` `2004` `2005` `2006` `2007` `2008`
   <chr>          <chr>  <chr>  <chr>  <chr>  <chr>  <chr>  <chr>  <chr>  <chr> 
 1 Stuttgart, St~ 35273~ 38408~ 39723~ 41115~ 40680~ 39624~ 42668~ 44532~ 42082~
 2 Böblingen      13867~ 15260~ 14664~ 14967~ 14478~ 12993~ 14904~ 17899~ 17315~
 3 Esslingen      14404~ 15465~ 14816~ 15216~ 15205~ 15237~ 16253~ 17230~ 17752~
 4 Göppingen      6000.~ 6048.~ 6099.~ 6216.~ 6158.~ 6077.~ 6366.~ 6695.4 6850.~
 5 Ludwigsburg    14657~ 15566~ 15731~ 15797~ 15851~ 16209~ 17243~ 18206~ 18580~
 6 Rems-Murr-Kre~ 10367~ 10434~ 10516~ 10416~ 10602~ 10642~ 11418~ 11588~ 11752~
 7 Heilbronn, St~ 5273.~ 5447.~ 5279.~ 4742.~ 4796.~ 4900.~ 5104.~ 5138.~ 5258.~
 8 Heilbronn, La~ 8453.~ 8816.~ 8749.~ 9212.~ 9436.~ 9842.~ 11048~ 11650~ 11985~
 9 Hohenlohekreis 3083.~ 3182.~ 3192.6 3208.~ 3302.0 3402.4 3630.~ 3894.~ 3941.~
10 Schwäbisch Ha~ 4503.~ 4525.~ 4716.~ 4634.3 4795.~ 5080.~ 5338.~ 5694.~ 5890.~
# i 445 more rows
# i 15 more variables: `2009` <chr>, `2010` <chr>, `2011` <chr>, `2012` <chr>,
#   `2013` <chr>, `2014` <chr>, `2015` <chr>, `2016` <chr>, `2017` <chr>,
#   `2018` <chr>, `2019` <chr>, `2020` <chr>, `2021` <chr>, `2022` <chr>,
#   `2023` <chr>
\end{verbatim}

\begin{Shaded}
\begin{Highlighting}[]
\CommentTok{\# Making raw\_econ into long format.}
\NormalTok{tidy\_econ }\OtherTok{\textless{}{-}} \FunctionTok{pivot\_longer}\NormalTok{(}\AttributeTok{data =}\NormalTok{ raw\_econ,}
             \AttributeTok{cols =} \SpecialCharTok{{-}}\NormalTok{Geo\_Labels,}
             \AttributeTok{names\_to =} \StringTok{"Time"}\NormalTok{,}
             \AttributeTok{values\_to =} \StringTok{"GDP Million EUR"}\NormalTok{)}
\end{Highlighting}
\end{Shaded}

\subsection{Demographic Data}\label{demographic-data}

Similarly as before, we do the same for the downloaded demographic
dataset from Eurostat, again using the ``read\_excel''-function. This
time we input the data into a function we have called ``raw\_demo'',
before we make it tidy and input that into ``tidy\_demo''.

\begin{Shaded}
\begin{Highlighting}[]
\CommentTok{\#Removing the metadata from the top.}
\NormalTok{raw\_demo }\OtherTok{\textless{}{-}} \FunctionTok{read\_excel}\NormalTok{(}\StringTok{"./Data/demo\_noFlag.xlsx"}\NormalTok{, }\AttributeTok{sheet =} \StringTok{"Sheet 1"}\NormalTok{, }\AttributeTok{range =} \StringTok{"A10:Z487"}\NormalTok{, }\AttributeTok{col\_types =} \StringTok{"text"}\NormalTok{)}
\end{Highlighting}
\end{Shaded}

\begin{verbatim}
New names:
* `TIME` -> `TIME...1`
* `TIME` -> `TIME...2`
\end{verbatim}

\begin{Shaded}
\begin{Highlighting}[]
\CommentTok{\#Dropping the first row to align time with years}
\NormalTok{  raw\_demo }\OtherTok{\textless{}{-}}\NormalTok{ raw\_demo[}\SpecialCharTok{{-}}\DecValTok{1}\NormalTok{,]}
  \FunctionTok{names}\NormalTok{(raw\_demo)[}\DecValTok{1}\NormalTok{] }\OtherTok{\textless{}{-}} \StringTok{"Geo\_Codes"}
  \FunctionTok{names}\NormalTok{(raw\_demo)[}\DecValTok{2}\NormalTok{] }\OtherTok{\textless{}{-}} \StringTok{"Geo\_Labels"}
\FunctionTok{print}\NormalTok{(raw\_demo)}
\end{Highlighting}
\end{Shaded}

\begin{verbatim}
# A tibble: 476 x 26
   Geo_Codes Geo_Labels  `2000` `2001` `2002` `2003` `2004` `2005` `2006` `2007`
   <chr>     <chr>       <chr>  <chr>  <chr>  <chr>  <chr>  <chr>  <chr>  <chr> 
 1 DE111     Stuttgart,~ 582443 583874 587152 588477 589161 590657 592569 593923
 2 DE112     Böblingen   362048 364987 367830 370337 371678 372113 372155 372228
 3 DE113     Esslingen   497826 500666 505340 509495 511564 513105 514245 514108
 4 DE114     Göppingen   256136 256792 257651 258488 258707 258492 257783 256967
 5 DE115     Ludwigsburg 495443 497764 503229 507043 509681 511830 513317 513998
 6 DE116     Rems-Murr-~ 407213 409296 412959 415764 416635 417463 417697 417609
 7 DE117     Heilbronn,~ 119526 119305 120163 120683 120705 121320 121613 121384
 8 DE118     Heilbronn,~ 317578 320955 324043 326229 327540 328866 329503 329979
 9 DE119     Hohenlohek~ 106930 107754 108920 109519 109755 109756 109718 109717
10 DE11A     Schwäbisch~ 184819 185728 186967 188229 188563 189041 189580 189346
# i 466 more rows
# i 16 more variables: `2008` <chr>, `2009` <chr>, `2010` <chr>, `2011` <chr>,
#   `2012` <chr>, `2013` <chr>, `2014` <chr>, `2015` <chr>, `2016` <chr>,
#   `2017` <chr>, `2018` <chr>, `2019` <chr>, `2020` <chr>, `2021` <chr>,
#   `2022` <chr>, `2023` <chr>
\end{verbatim}

\begin{Shaded}
\begin{Highlighting}[]
\CommentTok{\# Making raw\_demo into long format.}
\NormalTok{tidy\_demo }\OtherTok{\textless{}{-}} \FunctionTok{pivot\_longer}\NormalTok{(}\AttributeTok{data =}\NormalTok{ raw\_demo,}
             \AttributeTok{cols =} \FunctionTok{c}\NormalTok{(}\SpecialCharTok{{-}}\NormalTok{Geo\_Codes, }\SpecialCharTok{{-}}\NormalTok{Geo\_Labels),}
             \AttributeTok{names\_to =} \StringTok{"Time"}\NormalTok{,}
             \AttributeTok{values\_to =} \StringTok{"Population"}\NormalTok{)}
\end{Highlighting}
\end{Shaded}

\section{GDP Per capita:}\label{gdp-per-capita}

To calculate GDP per capita we have used the NUTS-3 column as a primary
key to join the tidied demographic and economic tables together. The
code chunk below joins the two datasets and adds a new column called
``GDP\_Capita'', calculated by multiplying the ``GDP Million
EUR''-column by a million and dividing it by the reported population in
the same year. We also add two more columns called ``Country'' and
``NUTS2'' by Extracting the first letters (which indicate country and
NUTS2-region) from the NUTS3-column.

\begin{table}

\caption{\label{tbl-tidyjoined}}

\centering{

\begin{Shaded}
\begin{Highlighting}[]
\CommentTok{\# Joining the two datasets}
\NormalTok{tidyjoined }\OtherTok{\textless{}{-}} \FunctionTok{left\_join}\NormalTok{(tidy\_demo, tidy\_econ, }\AttributeTok{by =} \FunctionTok{join\_by}\NormalTok{(Geo\_Labels, Time), }\AttributeTok{keep =} \ConstantTok{FALSE}\NormalTok{)}
\CommentTok{\# Mutating to add column for GDP per capita.}
\NormalTok{ tidyjoined }\OtherTok{\textless{}{-}}\NormalTok{ tidyjoined }\SpecialCharTok{\%\textgreater{}\%}
    \FunctionTok{mutate}\NormalTok{(}
    \StringTok{\textasciigrave{}}\AttributeTok{GDP Million EUR}\StringTok{\textasciigrave{}} \OtherTok{=} \FunctionTok{as.numeric}\NormalTok{(}\StringTok{\textasciigrave{}}\AttributeTok{GDP Million EUR}\StringTok{\textasciigrave{}}\NormalTok{),}
    \AttributeTok{Population =} \FunctionTok{as.numeric}\NormalTok{(Population),}
    \AttributeTok{GDP\_Capita =}\NormalTok{ (}\StringTok{\textasciigrave{}}\AttributeTok{GDP Million EUR}\StringTok{\textasciigrave{}} \SpecialCharTok{*} \DecValTok{1000000}\NormalTok{) }\SpecialCharTok{/}\NormalTok{ Population}
\NormalTok{  )}
\end{Highlighting}
\end{Shaded}

\begin{verbatim}
Warning: There were 2 warnings in `mutate()`.
The first warning was:
i In argument: `GDP Million EUR = as.numeric(`GDP Million EUR`)`.
Caused by warning:
! NAs introduced by coercion
i Run `dplyr::last_dplyr_warnings()` to see the 1 remaining warning.
\end{verbatim}

\begin{Shaded}
\begin{Highlighting}[]
\NormalTok{ tidyjoined }\OtherTok{\textless{}{-}}\NormalTok{ tidyjoined }\SpecialCharTok{\%\textgreater{}\%}
   \FunctionTok{mutate}\NormalTok{(}\AttributeTok{Country =} \FunctionTok{substr}\NormalTok{(Geo\_Codes, }\DecValTok{1}\NormalTok{, }\DecValTok{2}\NormalTok{), }\AttributeTok{.before =} \DecValTok{2}\NormalTok{)}
\NormalTok{ tidyjoined }\OtherTok{\textless{}{-}}\NormalTok{ tidyjoined }\SpecialCharTok{\%\textgreater{}\%}
   \FunctionTok{mutate}\NormalTok{( }\AttributeTok{NUTS2 =}\NormalTok{ (}\FunctionTok{str\_sub}\NormalTok{(Geo\_Codes, }\AttributeTok{start=} \DecValTok{1}\NormalTok{L, }\AttributeTok{end =} \DecValTok{4}\NormalTok{L)), }\AttributeTok{.before =} \DecValTok{2}\NormalTok{)}
\FunctionTok{print}\NormalTok{(tidyjoined)}
\end{Highlighting}
\end{Shaded}

\begin{verbatim}
# A tibble: 11,424 x 8
   Geo_Codes NUTS2 Country Geo_Labels         Time  Population `GDP Million EUR`
   <chr>     <chr> <chr>   <chr>              <chr>      <dbl>             <dbl>
 1 DE111     DE11  DE      Stuttgart, Stadtk~ 2000      582443            35274.
 2 DE111     DE11  DE      Stuttgart, Stadtk~ 2001      583874            38409.
 3 DE111     DE11  DE      Stuttgart, Stadtk~ 2002      587152            39723.
 4 DE111     DE11  DE      Stuttgart, Stadtk~ 2003      588477            41116.
 5 DE111     DE11  DE      Stuttgart, Stadtk~ 2004      589161            40680.
 6 DE111     DE11  DE      Stuttgart, Stadtk~ 2005      590657            39624.
 7 DE111     DE11  DE      Stuttgart, Stadtk~ 2006      592569            42668.
 8 DE111     DE11  DE      Stuttgart, Stadtk~ 2007      593923            44533.
 9 DE111     DE11  DE      Stuttgart, Stadtk~ 2008      597176            42082.
10 DE111     DE11  DE      Stuttgart, Stadtk~ 2009      600068            38337.
# i 11,414 more rows
# i 1 more variable: GDP_Capita <dbl>
\end{verbatim}

}

\end{table}%

\section{Descriptive analysis of the ``GDP per
capita''-table}\label{descriptive-analysis-of-the-gdp-per-capita-table}

\begin{Shaded}
\begin{Highlighting}[]
\CommentTok{\# Descriptive Analysis grouped by country code}
\FunctionTok{describeBy}\NormalTok{(}
\NormalTok{  tidyjoined,}
\NormalTok{  tidyjoined}\SpecialCharTok{$}\NormalTok{Country}
\NormalTok{)}
\end{Highlighting}
\end{Shaded}

\begin{verbatim}

 Descriptive statistics by group 
group: CH
                vars   n      mean        sd    median   trimmed       mad
Geo_Codes          1 624     13.50      7.51     13.50     13.50      9.64
NUTS2              2 624      4.00      1.86      5.00      4.08      1.48
Country            3 624      1.00      0.00      1.00      1.00      0.00
Geo_Labels         4 624    249.35    162.82    273.00    252.42    209.79
Time               5 624     12.50      6.93     12.50     12.50      8.90
Population         6 624 305461.88 319131.49 214911.50 242215.72 215700.51
GDP Million EUR    7 364  21507.00  25817.17  14450.17  16282.75  16671.43
GDP_Capita         8 364  65662.47  26094.69  59181.38  61016.74  13876.25
                     min       max     range  skew kurtosis       se
Geo_Codes           1.00      26.0      25.0  0.00    -1.21     0.30
NUTS2               1.00       7.0       6.0 -0.25    -1.37     0.07
Country             1.00       1.0       0.0   NaN      NaN     0.00
Geo_Labels          1.00     474.0     473.0 -0.18    -1.49     6.52
Time                1.00      24.0      23.0  0.00    -1.21     0.28
Population      14946.00 1579967.0 1565021.0  1.88     3.55 12775.48
GDP Million EUR   526.26  140799.6  140273.4  2.56     7.58  1353.19
GDP_Capita      30506.73  187233.7  156727.0  2.24     5.70  1367.73
------------------------------------------------------------ 
group: DE
                vars    n      mean        sd    median   trimmed      mad
Geo_Codes          1 9768    230.00    117.50    230.00    230.00   151.23
NUTS2              2 9768     26.54     11.78     27.00     26.43    16.31
Country            3 9768      2.00      0.00      2.00      2.00     0.00
Geo_Labels         4 9768    233.76    135.21    233.00    232.99   171.98
Time               5 9768     12.50      6.92     12.50     12.50     8.90
Population         6 9331 203245.16 235682.57 146751.00 166898.85 80591.17
GDP Million EUR    7 9202   7022.39  11430.89   4243.68   4979.97  2878.05
GDP_Capita         8 8765  32287.57  15350.99  28584.35  29874.51 10166.43
                     min       max     range skew kurtosis      se
Geo_Codes          27.00     433.0     406.0 0.00    -1.20    1.19
NUTS2               8.00      45.0      37.0 0.04    -1.30    0.12
Country             2.00       2.0       0.0  NaN      NaN    0.00
Geo_Labels          2.00     476.0     474.0 0.03    -1.18    1.37
Time                1.00      24.0      23.0 0.00    -1.20    0.07
Population      33264.00 3677472.0 3644208.0 8.35   100.64 2439.85
GDP Million EUR   806.67  197516.7  196710.0 7.35    72.01  119.16
GDP_Capita      10984.41  199296.2  188311.8 2.58    11.81  163.97
------------------------------------------------------------ 
group: HR
                vars   n      mean        sd    median   trimmed      mad
Geo_Codes          1 840    451.00     10.11    451.00    451.00    13.34
NUTS2              2 840     47.66      1.26     48.00     47.57     1.48
Country            3 840      3.00      0.00      3.00      3.00     0.00
Geo_Labels         4 840    279.03    135.22    300.00    284.21   169.02
Time               5 840     12.50      6.93     12.50     12.50     8.90
Population         6 595 198678.28 163448.17 141186.00 164221.88 52927.34
GDP Million EUR    7 504   2166.74   3226.54   1146.18   1459.15   727.21
GDP_Capita         8 315  10596.37   4177.19   9365.78  10035.89  2752.18
                     min       max     range  skew kurtosis      se
Geo_Codes         434.00    468.00     34.00  0.00    -1.21    0.35
NUTS2              46.00     50.00      4.00  0.41    -0.59    0.04
Country             3.00      3.00      0.00   NaN      NaN    0.00
Geo_Labels         44.00    471.00    427.00 -0.18    -1.20    4.67
Time                1.00     24.00     23.00  0.00    -1.21    0.24
Population      42469.00 809235.00 766766.00  2.46     5.96 6700.72
GDP Million EUR   237.32  25658.09  25420.77  4.08    18.25  143.72
GDP_Capita       4141.43  33381.85  29240.42  1.75     4.54  235.36
------------------------------------------------------------ 
group: IE
                vars   n      mean        sd    median   trimmed       mad
Geo_Codes          1 192    472.50      2.30    472.50    472.50      2.97
NUTS2              2 192     52.12      0.78     52.00     52.16      1.48
Country            3 192      4.00      0.00      4.00      4.00      0.00
Geo_Labels         4 192    266.88    134.78    254.50    270.30    200.15
Time               5 192     12.50      6.94     12.50     12.50      8.90
Population         6  96 607720.80 318232.23 472295.50 552867.55 169040.12
GDP Million EUR    7 185  30186.15  42352.46  13769.34  19981.87   9581.30
GDP_Capita         8  89  51718.66  41164.89  35327.62  43961.75  16962.68
                      min       max   range  skew kurtosis       se
Geo_Codes          469.00     476.0       7  0.00    -1.26     0.17
NUTS2               51.00      53.0       2 -0.22    -1.35     0.06
Country              4.00       4.0       0   NaN      NaN     0.00
Geo_Labels          51.00     455.0     404 -0.28    -1.14     9.73
Time                 1.00      24.0      23  0.00    -1.22     0.50
Population      286326.00 1499179.0 1212853  1.55     1.36 32479.44
GDP Million EUR   3765.33  248326.3  244561  2.83     8.57  3113.81
GDP_Capita       16456.08  193838.1  177382  1.73     2.03  4363.47
\end{verbatim}

Using the following code, we see that we have a total of 3838 NA-values
in our dataset. Most due to different ways of reporting demographic and
economic data, making the datasets hard to pair and leading to more
NA-values in the GDP\_Capita-column.

\begin{Shaded}
\begin{Highlighting}[]
\CommentTok{\#NA for all countries}
\FunctionTok{sum}\NormalTok{(}\FunctionTok{is.na}\NormalTok{(tidyjoined))}
\end{Highlighting}
\end{Shaded}

\begin{verbatim}
[1] 3838
\end{verbatim}

\begin{Shaded}
\begin{Highlighting}[]
\CommentTok{\# Simple statistics for all countries combined}
\FunctionTok{summary}\NormalTok{(tidyjoined)}
\end{Highlighting}
\end{Shaded}

\begin{verbatim}
  Geo_Codes            NUTS2             Country           Geo_Labels       
 Length:11424       Length:11424       Length:11424       Length:11424      
 Class :character   Class :character   Class :character   Class :character  
 Mode  :character   Mode  :character   Mode  :character   Mode  :character  
                                                                            
                                                                            
                                                                            
                                                                            
     Time             Population      GDP Million EUR      GDP_Capita    
 Length:11424       Min.   :  14946   Min.   :   237.3   Min.   :  4141  
 Class :character   1st Qu.: 105932   1st Qu.:  2544.7   1st Qu.: 22243  
 Mode  :character   Median : 150867   Median :  4220.9   Median : 28681  
                    Mean   : 212629   Mean   :  7715.8   Mean   : 33027  
                    3rd Qu.: 247409   3rd Qu.:  7596.0   3rd Qu.: 38111  
                    Max.   :3677472   Max.   :248326.3   Max.   :199296  
                    NA's   :778       NA's   :1169       NA's   :1891    
\end{verbatim}

\section{Regional GDP inequality - Using light levels as a predictor of
economic
development}\label{regional-gdp-inequality---using-light-levels-as-a-predictor-of-economic-development}

In this assignment we use reported GDP and demographic data from
Eurostat to determine regional inequality in a selection of countries,
but what can you do when regional income data isn't readily available?
The paper ``Regional inequality, convergence, and its determinants -- A
view from outer space'' by Christian Lessmann and André Seidel aimed to
find a new way of finding regional inequalities in areas without
economic data -- estimating regional income using satellite images of
nighttime light intensity.

Their method involved using luminosity data taken from meteorological
satellites from the U.S air force, and existing income data to estimate
a relationship between the two variables. They then used this estimate
to predict regional income for other regions where economic data was not
available, and to calculate inequality indicators such as the Gini
coefficient. The main takeaway from the study would be that yes -- it is
possible to use light as an indicator of GDP. Findings also showed that
for about 70\% of countries, regional gaps got smaller, while other
countries saw inequality grow. They also discovered an ``n-shaped'' link
between development and regional inequality: in early stages of growth
inequality is low, for mid-income regions it rises, before it falls
again in rich economies.

\section{Regional GDP inequality - Calculating the Gini
coefficient}\label{regional-gdp-inequality---calculating-the-gini-coefficient}

To calculate the Gini coefficient for our selected countries (weighted
for population) we first insert our data including GDP per capita
Table~\ref{tbl-tidyjoined} into a new function we have called
``ginigdp'' Table~\ref{tbl-ginigdp}. To be able to calculate the Gini
coefficient, we then have to remove NA-values from our dataset. This can
be done by using the ``na.omit''-function. The output is then grouped by
year (variable ``Time'') and region (NUTS2), and sent to a
summarise-function which includes our Gini calculation (done by the
``gini.wtd''-function). We have also included a count-column that shows
the amount of NUTS3-regions in each of the NUTS2-regions. This is to
provide clarity in case we get ``strange'' Gini-values like 0, which we
would represent ultimate equality. We will get this in all cases where
there is only one NUTS3-region per NUTS2-region. The code chunk below
does all this and prints the first 10 results.

\begin{table}

\caption{\label{tbl-ginigdp}}

\centering{

\begin{Shaded}
\begin{Highlighting}[]
\NormalTok{ginigdp }\OtherTok{\textless{}{-}}\NormalTok{ tidyjoined }\SpecialCharTok{\%\textgreater{}\%}
\NormalTok{  na.omit }\SpecialCharTok{\%\textgreater{}\%}
  \FunctionTok{group\_by}\NormalTok{(NUTS2, Time)}\SpecialCharTok{\%\textgreater{}\%}
  \FunctionTok{summarise}\NormalTok{( }\AttributeTok{Count =} \FunctionTok{n}\NormalTok{(),}
            \AttributeTok{gini =} \FunctionTok{gini.wtd}\NormalTok{(GDP\_Capita, }\AttributeTok{weights =}\NormalTok{ Population))}
\end{Highlighting}
\end{Shaded}

\begin{verbatim}
`summarise()` has grouped output by 'NUTS2'. You can override using the
`.groups` argument.
\end{verbatim}

\begin{Shaded}
\begin{Highlighting}[]
\FunctionTok{print}\NormalTok{(ginigdp)}
\end{Highlighting}
\end{Shaded}

\begin{verbatim}
# A tibble: 1,066 x 4
# Groups:   NUTS2 [52]
   NUTS2 Time  Count  gini
   <chr> <chr> <int> <dbl>
 1 CH01  2008      3 0.139
 2 CH01  2009      3 0.130
 3 CH01  2010      3 0.131
 4 CH01  2011      3 0.130
 5 CH01  2012      3 0.128
 6 CH01  2013      3 0.134
 7 CH01  2014      3 0.132
 8 CH01  2015      3 0.127
 9 CH01  2016      3 0.121
10 CH01  2017      3 0.128
# i 1,056 more rows
\end{verbatim}

}

\end{table}%

\section{Visualizing the Gini
coefficient}\label{visualizing-the-gini-coefficient}

In the sections below, we will use the newly created
``ginigdp''-function where all our gini-coefficients have been stored.
To do this, we first filter down to the specific country, before we send
the data to ggplot, group it by NUTS2-region and visualize using
geom\_point and geom\_line graphs.

\subsection{Switzerland}\label{switzerland}

\begin{Shaded}
\begin{Highlighting}[]
\CommentTok{\#Gini Switzerland}
\CommentTok{\#Plot for Switzerland }
\NormalTok{ginigdp }\SpecialCharTok{\%\textgreater{}\%}
  \FunctionTok{filter}\NormalTok{(}\FunctionTok{startsWith}\NormalTok{(NUTS2, }\StringTok{"CH"}\NormalTok{)) }\SpecialCharTok{\%\textgreater{}\%}
\FunctionTok{ggplot}\NormalTok{(}\AttributeTok{mapping =} \FunctionTok{aes}\NormalTok{(}\AttributeTok{x =}\NormalTok{ Time, }\AttributeTok{y =}\NormalTok{ gini, }\AttributeTok{colour =}\NormalTok{ NUTS2, }\AttributeTok{group =}\NormalTok{ NUTS2)) }\SpecialCharTok{+} 
  \FunctionTok{geom\_point}\NormalTok{(}\AttributeTok{mapping =} \FunctionTok{aes}\NormalTok{()) }\SpecialCharTok{+}
  \FunctionTok{geom\_line}\NormalTok{() }\SpecialCharTok{+}
  \FunctionTok{labs}\NormalTok{(}\AttributeTok{x =} \StringTok{"Year"}\NormalTok{, }\AttributeTok{y =} \StringTok{"Gini{-}Coefficient"}\NormalTok{, }\AttributeTok{title =} \StringTok{"Regional GDP Inequity in Switzerland"}\NormalTok{, }\AttributeTok{subtitle =} \StringTok{"By NUTS2{-}region"}\NormalTok{)}
\end{Highlighting}
\end{Shaded}

\begin{figure}[H]

\centering{

\pandocbounded{\includegraphics[keepaspectratio]{ass1msb104grp2_files/figure-pdf/fig-Switzerland-1.pdf}}

}

\caption{\label{fig-Switzerland}Regional GDP Inequity in Switzerland}

\end{figure}%

Figure~\ref{fig-Switzerland} shows low Gini-coefficients for each
region, with a tendency to stay below 0.20. There is seemingly a divide
into two groups one of which has a higher gini-coefficient than the
other. While the overall inequality seems to remain low and consistent,
we do see a divergence from the rest by CH03 who has seen growing
regional inequality the last 10+ years. This seems to be because CH031
Basel-Stadt has a much higher GDP per capita growth than the surrounding
areas. CH07 stays at 0 for the whole period due to only having one
region.

\subsection{Ireland}\label{ireland}

\begin{Shaded}
\begin{Highlighting}[]
\CommentTok{\#Gini Ireland}
\NormalTok{ginigdp }\SpecialCharTok{\%\textgreater{}\%}
  \FunctionTok{filter}\NormalTok{(}\FunctionTok{startsWith}\NormalTok{(NUTS2, }\StringTok{"IE"}\NormalTok{)) }\SpecialCharTok{\%\textgreater{}\%}
\FunctionTok{ggplot}\NormalTok{(}\AttributeTok{mapping =} \FunctionTok{aes}\NormalTok{(}\AttributeTok{x =}\NormalTok{ Time, }\AttributeTok{y =}\NormalTok{ gini, }\AttributeTok{colour =}\NormalTok{ NUTS2, }\AttributeTok{group =}\NormalTok{ NUTS2, )) }\SpecialCharTok{+} 
  \FunctionTok{geom\_point}\NormalTok{(}\AttributeTok{mapping =} \FunctionTok{aes}\NormalTok{()) }\SpecialCharTok{+} 
  \FunctionTok{geom\_line}\NormalTok{() }\SpecialCharTok{+}
  \FunctionTok{labs}\NormalTok{(}\AttributeTok{x =} \StringTok{"Year"}\NormalTok{, }\AttributeTok{y =} \StringTok{"Gini{-}Coefficient"}\NormalTok{, }\AttributeTok{title =} \StringTok{"Regional GDP Inequity in Ireland"}\NormalTok{, }\AttributeTok{subtitle =} \StringTok{"By NUTS2{-}region"}\NormalTok{)}
\end{Highlighting}
\end{Shaded}

\begin{figure}[H]

\centering{

\pandocbounded{\includegraphics[keepaspectratio]{ass1msb104grp2_files/figure-pdf/fig-Ireland-1.pdf}}

}

\caption{\label{fig-Ireland}Regional GDP Inequity in Ireland}

\end{figure}%

The irish graph seen in Figure~\ref{fig-Ireland} shows a bigger spread
and more ``movement'' than the swiss graph. We only have population data
from 2012 onwards, hence our starting point. As we can see in the graph,
IE05 and IE4 start out with similarly low Gini coefficients, both lying
around 0.1, whereas IE06, containing the capital Dublin, has a much less
even distribution of GDP per capita. If we take a look at
Table~\ref{tbl-tidyjoined}, we can see that for the period 2015-2017 no
GDP data was reported for the Mid-West and South-West NUTS3-regions,
leaving only one NUTS3-region remaining in the IE05 group, giving us a
Gini coefficient of 0,0 for those years. When GDP data returned in 2018,
IE053 (South-West) had grown in GDP per capita in a big way, pulling
away from the rest of the group and increasing the Gini coefficient. The
graph seems to show a trend towards slowly growing regional inequality
of development in Ireland.

\subsection{Germany}\label{germany}

\begin{Shaded}
\begin{Highlighting}[]
\CommentTok{\#Gini Germany}
\NormalTok{ginigdp }\SpecialCharTok{\%\textgreater{}\%}
  \FunctionTok{filter}\NormalTok{(}\FunctionTok{startsWith}\NormalTok{(NUTS2, }\StringTok{"DE"}\NormalTok{)) }\SpecialCharTok{\%\textgreater{}\%}
\FunctionTok{ggplot}\NormalTok{(}\AttributeTok{mapping =} \FunctionTok{aes}\NormalTok{(}\AttributeTok{x =}\NormalTok{ Time, }\AttributeTok{y =}\NormalTok{ gini, }\AttributeTok{colour =}\NormalTok{ NUTS2, }\AttributeTok{group =}\NormalTok{ NUTS2)) }\SpecialCharTok{+} 
  \FunctionTok{geom\_point}\NormalTok{(}\AttributeTok{mapping =} \FunctionTok{aes}\NormalTok{()) }\SpecialCharTok{+} 
  \FunctionTok{geom\_line}\NormalTok{() }\SpecialCharTok{+}
  \FunctionTok{theme}\NormalTok{(}\AttributeTok{axis.text.x =} \FunctionTok{element\_text}\NormalTok{(}\AttributeTok{angle =} \DecValTok{45}\NormalTok{)) }\SpecialCharTok{+}
  \FunctionTok{labs}\NormalTok{(}\AttributeTok{x =} \StringTok{"Year"}\NormalTok{, }\AttributeTok{y =} \StringTok{"Gini{-}Coefficient"}\NormalTok{, }\AttributeTok{title =} \StringTok{"Regional GDP Inequity in Germany"}\NormalTok{, }\AttributeTok{subtitle =} \StringTok{"By NUTS2{-}region {-} full graph"}\NormalTok{)}
\end{Highlighting}
\end{Shaded}

\begin{figure}[H]

\centering{

\pandocbounded{\includegraphics[keepaspectratio]{ass1msb104grp2_files/figure-pdf/fig-Germany-1.pdf}}

}

\caption{\label{fig-Germany}Regional GDP Inequity in Germany - Full
graph}

\end{figure}%

Germany has been very consistent at reporting data and we have data for
the full period, but as Figure~\ref{fig-Germany} shows, we have a little
bit of an information overload on our hands. Germany consists of up to
38 NUTS2 regions which is crowding the graph, making it very hard to
read. Our dataset gives a spread from 0 to 0.4 in gini-coefficient. To
combat the information overload we have chosen to extract 10 random
regions to get a better picture of Germany's regional inequalities.

\begin{Shaded}
\begin{Highlighting}[]
\CommentTok{\# Picking out 10 different regions for Germany}
\CommentTok{\# Set.seed to not get random every time one runs the codes}
\FunctionTok{set.seed}\NormalTok{(}\DecValTok{123}\NormalTok{) }

\NormalTok{ginigdp }\SpecialCharTok{\%\textgreater{}\%}
  \FunctionTok{filter}\NormalTok{(}\FunctionTok{startsWith}\NormalTok{(NUTS2, }\StringTok{"DE"}\NormalTok{)) }\SpecialCharTok{\%\textgreater{}\%}
  \FunctionTok{group\_by}\NormalTok{(NUTS2) }\SpecialCharTok{\%\textgreater{}\%}
  \FunctionTok{summarise}\NormalTok{() }\SpecialCharTok{\%\textgreater{}\%}
  \FunctionTok{sample\_n}\NormalTok{(}\DecValTok{10}\NormalTok{) }\SpecialCharTok{\%\textgreater{}\%}
  \FunctionTok{pull}\NormalTok{(NUTS2) }\OtherTok{{-}\textgreater{}}\NormalTok{ sampled\_regions}

\NormalTok{ginigdp }\SpecialCharTok{\%\textgreater{}\%}
  \FunctionTok{filter}\NormalTok{(NUTS2 }\SpecialCharTok{\%in\%}\NormalTok{ sampled\_regions) }\SpecialCharTok{\%\textgreater{}\%}
  \FunctionTok{ggplot}\NormalTok{(}\AttributeTok{mapping =} \FunctionTok{aes}\NormalTok{(}\AttributeTok{x =}\NormalTok{ Time, }\AttributeTok{y =}\NormalTok{ gini, }\AttributeTok{colour =}\NormalTok{ NUTS2, }\AttributeTok{group =}\NormalTok{ NUTS2)) }\SpecialCharTok{+}
  \FunctionTok{geom\_point}\NormalTok{() }\SpecialCharTok{+}
  \FunctionTok{geom\_line}\NormalTok{() }\SpecialCharTok{+}
  \FunctionTok{theme}\NormalTok{(}\AttributeTok{axis.text.x =} \FunctionTok{element\_text}\NormalTok{(}\AttributeTok{angle =} \DecValTok{45}\NormalTok{)) }\SpecialCharTok{+}
  \FunctionTok{labs}\NormalTok{(}\AttributeTok{x =} \StringTok{"Year"}\NormalTok{, }\AttributeTok{y =} \StringTok{"Gini{-}Coefficient"}\NormalTok{, }\AttributeTok{title =} \StringTok{"Regional GDP Inequity in Germany"}\NormalTok{, }\AttributeTok{subtitle =} \StringTok{"By NUTS2{-}region {-} randomly selected."}\NormalTok{)}
\end{Highlighting}
\end{Shaded}

\begin{figure}[H]

\centering{

\pandocbounded{\includegraphics[keepaspectratio]{ass1msb104grp2_files/figure-pdf/fig-DErandom-1.pdf}}

}

\caption{\label{fig-DErandom}Regional GDP Inequity in Germany - Random
selection}

\end{figure}%

Figure~\ref{fig-DErandom} takes 10 randomly selected NUTS2 regions and
shows the same spread as the one with all regions. The graph
Figure~\ref{fig-Germany} containing all regions from Germany showed one
region with a Gini of 0,4 while most other regions lie between 0,1 and
0,2. The pattern seems to be the Gini coefficient remains stable over
time, but we do have two significant ``jumps''. DEB3 shows a significant
jump from 0,24 in 2020 to 0,32 in 2021. This seems to be because DEB35
Mainz doubles in reported GDP, with their population remaining stable.
The other regions in the group do not see a similar jump, hence
inequality grows. The other jump can be seen in DE80, where we have a
jump from 0,017 (extremely low) in 2010 to 0,083 in 2011. If we refer to
the tidyjoined-table Table~\ref{tbl-tidyjoined} , we see that population
data was only reported for two of the NUTS3-regions in DE80 until 2011,
giving us artificially low data for the preceeding period. The Gini
coefficient for DE80 is therefore not comparable between the periods
before 2010 and after 2011. DE60 only contains one NUTS3-region
(Hamburg) and is therefore stable at 0.

\subsection{Croatia}\label{croatia}

\begin{Shaded}
\begin{Highlighting}[]
\CommentTok{\#Gini Croatia}
\NormalTok{ginigdp }\SpecialCharTok{\%\textgreater{}\%}
  \FunctionTok{filter}\NormalTok{(}\FunctionTok{startsWith}\NormalTok{(NUTS2, }\StringTok{"HR"}\NormalTok{)) }\SpecialCharTok{\%\textgreater{}\%}
\FunctionTok{ggplot}\NormalTok{(}\AttributeTok{mapping =} \FunctionTok{aes}\NormalTok{(}\AttributeTok{x =}\NormalTok{ Time, }\AttributeTok{y =}\NormalTok{ gini, }\AttributeTok{colour =}\NormalTok{ NUTS2, }\AttributeTok{group =}\NormalTok{ NUTS2)) }\SpecialCharTok{+} 
  \FunctionTok{geom\_point}\NormalTok{(}\AttributeTok{mapping =} \FunctionTok{aes}\NormalTok{()) }\SpecialCharTok{+} 
  \FunctionTok{geom\_line}\NormalTok{() }\SpecialCharTok{+}
  \FunctionTok{theme}\NormalTok{(}\AttributeTok{axis.text.x =} \FunctionTok{element\_text}\NormalTok{(}\AttributeTok{angle =} \DecValTok{45}\NormalTok{)) }\SpecialCharTok{+}
  \FunctionTok{labs}\NormalTok{(}\AttributeTok{x =} \StringTok{"Year"}\NormalTok{, }\AttributeTok{y =} \StringTok{"Gini{-}Coefficient"}\NormalTok{, }\AttributeTok{title =} \StringTok{"Regional GDP Inequity in Croatia"}\NormalTok{, }\AttributeTok{subtitle =} \StringTok{"By NUTS2{-}region."}\NormalTok{)}
\end{Highlighting}
\end{Shaded}

\begin{figure}[H]

\centering{

\pandocbounded{\includegraphics[keepaspectratio]{ass1msb104grp2_files/figure-pdf/fig-Croatia-1.pdf}}

}

\caption{\label{fig-Croatia}Regional GDP Inequity in Croatia}

\end{figure}%

The graph of Croatia Figure~\ref{fig-Croatia} shows us a bit of a
different picture from the other countries. Demographic data seems to be
missing in the ``ginigdp''-table for all regions but HR03 before 2013,
hence the early start for HR03. Croatia has had its NUTS regions change
several times between 2000 to 2023, and the data available from Eurostat
contains both the old and new definitions, making it hard to pair the
two datasets. For instance, the capital Zagreb appears as both HR041 and
HR050 using different NUTS-definitions. Because we joined the two
Eurostat-datasets by NUTS3-code, we end up getting a graph that looks
incomplete, but this appears to be the best way to pair the available
datasets with eachother while still limiting room for error.

If we take a look at the calculated Gini coefficients in
Table~\ref{tbl-ginigdp} , we see that we get very low values for all
available croatian regions. This could usually mean that we have very
few NUTS3-regions per group, but if we take a look at the data, we see
that this is not the case. HR06 for instance, has 5 NUTS3-regions inside
of it, all with similar GDP per capita. The fact that all of Croatia's
regions have such a low Gini coefficient could indicate that economic
development is evenly spread inside the NUTS2-regions. HR05 only
contains the capital Zagreb, and is therefore shown as 0.

\section{Implications of our
findings}\label{implications-of-our-findings}

In this assignment we have calculated Gini coefficients inside each
available NUTS2-region using Eurostat data for Ireland, Germany, Croatia
and Switzerland. The calculated variations in the Gini coefficient tell
us something about the variations of gdp per capita inside each
NUTS2-region. A high Gini could indicate that we have a case of a highly
productive city-region inside a greater region containing a lot of less
productive land.

Overall, the the calculated Gini coefficients seem to be stable over
time, except for a few outliers as commented on previously. This could
mean that economic development in our selected countries is mostly
stable, and not especially concentrated in a handful of highly
productive cities inside larger regions.

Our findings do however not entirely dismiss the idea that growth and
economic development is mostly centered around cities. If we take a look
at the data for Croatia in Table~\ref{tbl-tidyjoined}, we can see that
Zagreb has been designated as its' own region, with a GDP per capita
much higher than the other regions in Croatia. Had Zagreb instead been a
part of any of the other regions, the calculated Gini coefficient would
have been much higher than what Figure~\ref{fig-Croatia} showed. The
same is the case for Berlin, Hamburg and many other highly productive
cities, but because they are designated to their own NUTS2-regions, they
end up as ``blind spots'' for this specific assignment and end up with a
Gini of 0. To further examine this theory, it would be interesting to
look at the calculated Gini coefficients based on variations of GDP per
capita per NUTS2 region, which might be able to pick up these
disparities.

\section{The use of AI}\label{the-use-of-ai}

In this assignments there was used AI to confirm through controlling
questions and constructive judging the text and the codes used, to
provide a constructive feedback. The AI was used as a sparring partner
to help with the wording of the writing and testing of the codes to
provide an explanation of how each function in the code works. The
software of AI we used was ChatGPT and the model was 3.5 which is the
free version. This is a quick and basic model, which means that the text
produced by this type of model must be checked. It was only used to
check the functioning of our codes and recommend spellings of text.




\end{document}
